% Items are shown in "main.bib" file\\

% \noindent An article \cite{anarticle} - The required fields here are author, title, journal, and year; all others can be omitted if you do not have the corresponding information. Note as well that author names should be separated by “and”.\\
% A book \cite{abook} - The required fields are author, title, publisher, and year.\\
% A series \cite{bookseries} - Series are cited as books with an additional field.\\
% Someone's thesis \cite{thesis} - The required fields are author, title, publisher, and year. You may also
% cite master’s theses using the \textit{mastersthesis} entry type.\\
% Some technical report \cite{report} - The required fields are author, title, publisher, and year.\\
% A collection \cite{collection} - The required fields are author, title (of the article within the book), book-
% title (name of the book containing the article), and year.\\
% Visited website \cite{website} - For websites, note the specific formatting of the howpublished field. Be
% sure to include the date of access as shown. Also, keep in mind that certain
% characters common in urls, such as underscores, must be escaped (preceded
% by a backslash, ex.\\
% Accepted for publication \cite{acceptedpub} - To show that an article has been accepted for publication but not yet
% published, include the line note = {(in press)}.\\
% Submitted for publication \cite{unpub} - cited as published with an additional field.\\
% Not published \cite{notpub} - If a manuscript has not yet been published or submitted for publication,
% you may still cite it. Include a note to this effect.\\
% Conversation \cite{conv} - Occasionally, an important item that does not exist in the literature will
% be discussed with your mentor. When this happens, you may cite the conversation with your mentor.