
All reviewed software offers a feature which is significant to detect malware and ransomware. Namely packet logging, or packet capture. There are many ways this can be done. Either by having NIDS inline, as a HIDS connected to a port in a switch in promiscuous mode etc. There are also many variations over how the detection, threat hunting, is performed in practice. The common theme is that Static Analysis is performed on captured packet data, or binary extracted from packet data. OSSEC, Snort or Zeek can be all be used as packet logger. It is with the captured packet data threat hunting is performed on. This can be performed using different Machine/Deep Learning algorithms and methodologies. One deep learning approach used labelled data set of malware, which has been transformed to a 32X32 image matrix data. By applying a Support Vector Machine (SVM) algorithm, the method has successfully classified the training data set with up to 84.92\%\cite{322221656}. More recent experiment managed to increase machine learning aided detection approach up to 99.9\% accuracy\cite{358776502}. The improvement has been achieved using a combination of machine learning algorithm, an ensemble learning approach. Along with a properly prepared training data set to minimize noise, over and underfitting, combining classes to minimise imbalanced data set etc.

Acquiring the software required to implement a defensive security solution, IDS, IPS, SIEM etc., is not necessarily a huge investment. As there are many Open-Source software which are near equivalent with proprietary or payed solutions. The software reviewed were Open-Source, while Snort has both Open-Source and commercial version. They all offer similar technical specs, features and functionalities. Snort has the advantage of full vendor support of Cisco Systems. A large Fortune500 company, and a leading network security vendor, provides Snort the trust and confidence in troves.The basic Snort installation provides a basic rule set. A version of the rule set is publicly available for free, as provided and maintained by the researchers from Cisco Talos Intelligence Group, whom are the official content creators for Snort. Cisco Systems, together with a large and engaged Open Source community, provides a freely available viable version of the product for non-enterprise or SMB users. Where large investments can be required is with eventual dedicated hardware, network devices, servers or sensors. Security is also a knowledge based, labor intensive undertaking. In-house resource, external vendor support or outsourced service agreement will have recurring cost. Designing a network that lends to maintainability, security monitoring etc. requires a competent network architect. Not to mention the fact that the monitoring server, network sensors, log servers etc. must be secured as well.

As such, all of the software reviewed is a viable solution, in terms of technical performance, efficacy, and cost effectiveness. Zeek and Snort are very capable and customisable. They also integrate well with other systems used in defensive security. OSSEC on the other-hand comes bundled with AlienVault OSSIM, the Open-Source solution of AT\&T's SIEM offering. Installing OSSEC as stand alone performs as good as the other software reviewed. But installing is as part of OSSIM bundle may be easier to deploy, maintain and provide a more rounded defensive security solution.

