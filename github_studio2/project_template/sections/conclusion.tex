Security best-practice rely on a holistic and layered approach...

\begin{followup}
    The following items are statements from the abstract that we must address:
    
    \begin{itemize}
        \item Review how these open source software addresses the identified network threats
        \item IDS features, capabilities, advantages and limitations will be compared.
        \item How each system fare with regards to manageability, cost effectiveness and return of investment
        \item Determine which IDS/IPS are most suitable for SMBs
    \end{itemize}
    
\end{followup}


\begin{followup}[to-do]
    \begin{itemize}
        \item Provide evidence for security best-practices to further elaborate and support the statement.
        \item Provide concrete examples where IDS and IPS are ineffective against malware and ransomware.
    \end{itemize}
\end{followup}

With all the limitations inherent to IDS and IPS taken into consideration. The most effective, cost effective and manageable systems is...

\begin{followup}[to-do]
    \begin{itemize}
        \item Insert name of IDS/IPS here when all the review has been conducted.
    \end{itemize}
\end{followup}


The IDS an IPS systems must also be protected along with the logs and packet capture data they store.