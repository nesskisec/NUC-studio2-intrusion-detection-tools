\subsection{A brief background and history}

IDS and IPS are a category of network tools or systems to detect and prevent malicious network activities. One of the earliest network based security defenses was firewall. Firewalls limited filter packets. Thus were IDS' introduced, which had the capability to inspect and validate individual packet and validate signatures of different indicators of compromise (IOC). Which are able to alert the administrators of positively identified malicious communication. Later, the integration of detection and prevention (IDS and IPS) in a single systems or devices like next-generation firewalls, further enhancing the security postures of organizations where these systems are deployed.

\subsection{Evolving Threat Landscape and Countermeasures}

IDS were initially installed on a host server, then specific network IDS devices gained even more popularity by providing IDS and IPS functionality in an integrated and more conveniently deployable and manageable unit.
\newpage
\subsection{Detection and Prevention for SMBs'}

The scope of this state-of-art review are open-source IDS systems. Though occasionally, this paper will look at how an IDS relates to a larger tool chain, systems or platforms like SIEM's. The wider defensive security solution. This is inevitable as security relies on a layered approach, and reliance on a single tool, method or policy is bad practice. It is also relevant to assess an IDS as part of a companies infrastructure roadmap as the company scales up. SIEMs' are typically deployed in larger Enterprise IT environments. SIEMs takes center the stage in a companies Security Operations Center (SOC). These are often complex systems due to business process integration, or systems integration with IT Service Management tools like SolarWinds, ServiceNow or Jira, to name a few. All of the above drives cost of ownership up. Some may opt to subscribe to SOC as services to reduce cost.


\subsection{Addressing SMBs' most prevalent network threat with IDS}

Malware and ransomware, is identified in the Norsis \cite{Norsis2021} threat report for 2021 as a prevalent threat. Malware and Ransomware pose difficult challenge as they apply various techniques to obfuscate their existence on the host. With Polymorphic and Metamorphic code, malware are able to hide while at rest on the host and when initiating communication on the network. And they can also piggy-back on trusted network sessions. Rendering the binary signatures used to identify them unusable using above mentioned techniques. On the other hand IDS and IPS main function is to detect, alert and prevent threats that are traversing the network. Therefore it is crucial that IDS and IPS are deployed together with other security tools and systems to be able to maximize its utilization. Larger network may also benefit from properly segmented and architecturally designed with security in mind. As always, a layered security approach must be applied.\\

\newpage