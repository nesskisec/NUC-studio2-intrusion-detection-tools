\subsubsection{Reliable and Well Supported And Well Supported}

Along with technical specs, features and functionalities, the product and vendor reliability and confidence define the key factors that are assessed and evaluated prior to implementation and deployment. Specially for a crucial component that goes into a security system. Having Cisco Systems, a large Fortune500 company, and a leading network security vendor, provides Snort the trust and confidence in troves. For example, the backbone of Snort is its Rules which enables Snort to detect malicious network sessions. The basic Snort installation provides a basic set of rules. This rules are publicly available for free as provided and maintained by the researchers from Cisco Talos Intelligence Group, whom are the official content creators for Snort. Cisco Systems, together with a large and engaged Open Source community, provides a freely available vailable version of the product for non-enterprise or SMB users.

Snort's 3 main features or mode of operation, as ofted cited in many Snort tutorial blog post entries, how-to videos on YouTube and Wikipedia are; Packet Sniffer, Packet Logger and Network Intrusion Detection System mode\cite{Wikipedia2022_Snort}. However, Cisco Systems themselves often refer to Snort as an Intrusion Prevention System in their documentation\cite{Ciscco_Snort3_Configuration_Guide_v7} and Cisco Talos Intelligence Group's YouTube video\cite{TIG2020_Snort_101}. As a packet sniffer, it is capable to capture, interpret and process packets akin to Wireshark and tcpdump. As a packet logger Snort can capture network packets (without analyzing) and store them for further analysis.As an Intrusion Detector, Snort has the ability to log events, trigger alerts and other systems it can integrate with like SIEMs. As an Intrusion Prevention System, it can drop packets from the network or trigger other systems that are able to execute preventive measures.\\

\subsubsection{IDS/IPS - Installation}

Lets establish a scenario where we want to address the issue of general intrusion and more specifically malware and ransomware. In this scenario we want to alert malicious network intrusion. At the same time detect malware prevent file extraction from a specific on-site file server.

One simple way to achieve this objective is to implement Snort as an inline NIDS/NIPS inside the SMB network, right behind a firewall. A dedicated AT\&T OSSIM server as SIEM alert and monitoring dashboard. And a dedicated server to run malware analysis on automated with Python scripts and libraries.

The Snort NIDS/NIPS can be configured to perform packet logging regularly with specific interval. This is necessary to capture data that can be used for malware analysis. A static analysis, automated with Python scripts utilizing OpenSource libraries for Machine and Deep Learning, read Windows Portable Executables (PE) etc.

There are different methods to apply Machine Learning to perform malware analysis as described in Alessandro Parisi's "Hands-On Artificial Intelligence for CyberSecurity"\cite{Parisi2019} while A YouTube video by Motasem Hamdan describes basic installation of Snort\cite{YouTube2022_HamdanM}\\