\textbf{Privacy \& GDPR}\\
\\
A considerable complication concerning the utilization of IDS tools due to laws in some countries surrounding privacy. This due to the GDPR laws which companies regardless of their size could be fined a substatial amount for not being compliant with. That could lead to a significant financial blow or in worst case bankrupcy for some companies. One of the latest cases from Spain showing that Google got fined 10 million Euro, for violation of the GDPR rules.\cite{Enforcementtracker.com} If this fine was given a smaller company than Google, for the same violation, this could hurt the company tremendously. Although a smaller company possibly surviving the financial blow, the reputational damage could lead to more financial loss.\\
\\
Regardless of IDS systems aid to improve the security by detecting vulnerabilities and attacks to prevent compromising systems, not all the information are supposed to be seen or supervised by everyone on the same system or network. One solution for this issue could be Encryption, but as a survey from Khraisat, Gondal, Vamplew and Kamruzzaman state that the encrypted traffic makes it difficult to detect attacks. \cite{Khraisat2019} This makes it complicated due to the GDPR rules, because if the normal traffic on a system is not encrypted, this is available in plain text for anyone not only on the system but to intruders as well. To restrict all parties within a company having access to view the traffic from IDS, a user role and permissions could be something to look into further for this matter.\\
\\
\textbf{Anonymisation}\\
\\
\textbf{Data Confidentiality}\\
\\
\textbf{Reputation}\\

\begin{question}[linking ethics to GDPR]

    Is this useful? Shall we keep it?
    
\end{question}
    
    
From a small business owner, or an service provider point-of-view, are there any ethical issues or ramifications that must be concidered when deploying Intrusion Detection Systems? The short answer is; Yes there are some ethical considerations that must be addressed when deploying IDS. This question can be approached as formal techinicallity or in a more philosophical manner.

Then there is a philosophical approach. This is perhaps taken into concideration in an unconsious level in a daily basis which has to do with subjective internal values.

From a philosophical perspective, we can regard ethics as a framework. A tool to help us navigate and resolve conflicting ideas and value judgement. From where Kantianism, Utilitarianism and Contractarianism lended their specific views on ethics, which has helped form and define how our modern society is governed and policed \cite{Courtland2017}.

Ethics, although primarily a subjective set of beliefs used. It does manifests as individual values our policy makers adheres to. Whom then forms a common cultural set of values and beliefs in their individual arenas and communitities. Which then translates to the cultural norms influencing and defining our politics, rules and regulations. And by that our daily lives and businesses.

Thus, one can say that ethics are already built into the common set of laws and regulations business, and individuals alike, are protected by and must adhere to. This is how ethics are formalized and applied in the "real world". Directly influencing the business domain. And in the case of CyberSecurity, where IDS and monitoring logs captures user information, GDPR compliance provisions ethical concerns to be addressed \cite{Haberkorn2019}.

How GDPR affects local business can be quite complicated. Where the business is located determines how GPDR affects a business and what GDPR compliance are required. GDPR is a regulation for EU member states and its citizens. Therefore, any business that caters towards EU citizens must comply. While businesses that processes or store personal identifiable informaton\cite{EuropeanCommission-PII} (PII) can technically be required to comply\cite{Wolford-GDPR_outside_EU}.

A business residing in Norway however must comply with national regulations concerning personal data as mandated by the Norwegian Data Protection Authority (DataTilsynet). Although Norways has its own national regulations for personal data, it does comply with the GDPR regulations is a great degree and can be regarded as proxy implementation of GDPR.Datatilsynet may also require a dedicated "Data Proctection Officer" depending on the nature of the business\cite{Datatilsynet2019_personvernombud}.

\begin{notes}[Citation needed]
    Lookup details at datatilsynet.no\\

    https://www.datatilsynet.no/en/\\

    https://www.datatilsynet.no/globalassets/global/dokumenter-pdfer-skjema-ol/regelverk/veiledere/dpia-veileder/dpialist280119.pdf\\

\end{notes}

In practice, a SMB in Norway that implements IDS which collects and stores PII, must comply with Datatilsynets regulations. At the minimum, the initial "Data Protection Impact Assessment" must be conducted. Furthermore, there some regulations and compliance which are sector specific, ekomforskriften for electronic communications services \cite{Lovdata2021_ekomforskriften} and finansforetaksloven \cite{Lovdata2016_finansforetaksloven} for financial institutions. Also important to mention is the special
regulation \cite{Lovdata2018_Sikkerhetsloven} for businesses that provides services to certain public institutions or institutions with national import.

From a business and a service provider point of view, the there are are few tenets which must be complied: Each individual users must able to request for their PII to be purged...