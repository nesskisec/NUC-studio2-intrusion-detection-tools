\textbf{Privacy \& GDPR}\\
\\
A considerable complication concerning the utilization of IDS tools due to laws in some countries surrounding privacy. This due to the GDPR laws which companies regardless of their size could be fined a substatial amount for not being compliant with. That could lead to a significant financial blow or in worst case bankrupcy for some companies. One of the latest cases from Spain showing that Google got fined 10 million Euro, for violation of the GDPR rules.\cite{Enforcementtracker.com} If this fine was given a smaller company than Google, for the same violation, this could hurt the company tremendously. Although a smaller company possibly surviving the financial blow, the reputational damage could lead to more financial loss.\\
\\
Regardless of IDS systems aid to improve the security by detecting vulnerabilities and attacks to prevent compromising systems, not all the information are supposed to be seen or supervised by everyone on the same system or network. One solution for this issue could be Encryption, but as a survey from Khraisat, Gondal, Vamplew and Kamruzzaman state that the encrypted traffic makes it difficult to detect attacks. \cite{Khraisat2019} This makes it complicated due to the GDPR rules, because if the normal traffic on a system is not encrypted, this is available in plain text for anyone not only on the system but to intruders as well. To restrict all parties within a company having access to view the traffic from IDS, a user role and permissions could be something to look into further for this matter.\\
\\
\textbf{Anonymisation}\\
\\
\textbf{Data Confidentiality}\\
\\
\textbf{Reputation}\\
