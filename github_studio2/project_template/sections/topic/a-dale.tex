\subsubsection{SNORT}

\begin{notes}[Individual subtopic - SNORT]

    Work on-going:

    \begin{itemize}
        \item Gathering information about implementation and deployment examples that can be applied to SMBs.
        \item Gathering concret implementation example where SNORT is used in conjucntion with ML and DL to address polymorphic and metamorphic malware.\\
    \end{itemize}
    
\end{notes}


Intrusion detection is experimental (A novel approach to intrusion detection using
SVM ensemble with feature augmentation, Jie Gu et al, 2019) and (Hands-On Artifical Intelligence for CyberSecurity, Parisi A, 2019, p 22)


Malware Analysis objectives:

- Distinguish malicious binary files from legitimate files.
- Automate the preparatory phase of malware analysis; triage.
- The analyst must understand the logic by which the Machine Learning, and Deep Learning tools apply; for finetuning, knowing what method or algorithm to address relevant to the task at hand. Properly assess the results and how to adapt.

What characterizes an efficient Malware detection software:

- A quick and successful preliminary screening of possible malicious binary.
- An analyst that can quickly verify and alert about the malicious binary.
- Adaptive to new threats and their contextual changes

What can be malicious files:

- All files, even non-executable files such as a .PDF, .TXT, .JPG etc. (Hands-On Artifical Intelligence for CyberSecurity, Parisi A, 2019, p 112)

Modes of malware analysis:

- Static analysis
- Dynamic analysis


What are malwares:

- Trojans
- Botnets
- Downloaders
- Rootkits
- Ransomware
- APT
- Zero Days