
\subsubsection{Snort}

As of 2022, there are 2 major versions of Snort installation base, namely Snort 2.X and Snort 3.X. There are 3 key, and perhaps the most noticeable, differences between Snort 2 and Snort 3. It has to do with performance optimization, detection rules and software distribution. With regards to performance, Snort 3 code base has been refactored and optimized and it also now support multi-threading. Enabling Snort 3 to perform packet analysis multi-threaded. Snort 3 has also improved Rules syntax and parsing using LUA programming language \cite{Snort2020_MunshawJ}. Since Snort3 was first introduced not too long ago, in 2021-01-19, the Snort 2.X installation base can be expected to have a long tail. Even though Snort 3.X introduced many meaningful improvements, Snort 2.X still do function and is still a supported product. Both by the its vendor, other service providers, the Open Source community and its users. For instance, the publicly available rules set are still being maintained and updated for both Snort 2.X and 3.X. 

Network packets are processed by Snort in 5 stages. Stage 1 is the is called the "Packet", where the network packet is accepted by the system for processing. Stage 2 is the "Decoder" which determines what protocol is used and performs protocol header analysis where is can identify malformed header. This stage prepares the packet for further inspection. Stage 3 is the "Preprocessor" where the packet is normalized using Snort plug-ins. Plug-ins that allow Snort to process specific type of network traffic. Pre-processing entails eliminating anti IDS and IPS techniques and re-formats the packets for easier analysis where Snort rules can be applied. Stage 4 is where the "Detection" is performed. This stage makes use of the rule sets available and applies it using a decision tree to determines if a packet is malicious. This stage has the ability to send an alert or prevent an intrusion by accepting or dropping the packet. Stage 5 is the "Log and Verdict" stage that performs the traffic logging. The logs can be forwarded to a log processor like Splunk or compatible SIEM. The article "Deep Packet Inspection for Intrusion Detection Systems: A Survey" by Tamer AbuHmed et al. thoroughly describes the intrusion detection process and expands in detail the algorithms used by the system \cite{Tamer_et_al2008_Deep_Packet_Inspections}.\\
\\
\textit{Snort - Detecting Malware and Ransomware}

It has to be stated that AI and Machine Learning within the problem domain of intrusion detection continuous to be regarded an experimental\cite{Parisi2019} endeavour at this point in time. In that there seem to be no established superior method, algorithm or process to automatically calibrate detection rules and algorithm to be used in IDS and IPS. However, studies has indeed proven that a Machine/Deep Learning is a viable solution. And that Machine and Deep Learning methodologies can indeed be used with Snort packet captures for analysis, albeit not in real-time. A real-time solution will require both high bandwidth network, high capacity storage and high compute capability. Snort, together with other Open Source Software can be utilized to automate the capture, detection and alert functionalities in near real time. A HIDS/HIPS solution can be implemented with Snort, OSSIM, a few Python Libraries and some custom scripts. A similar NIDS/NIPS is possible provided network, storage and compute requirements are met.
\\