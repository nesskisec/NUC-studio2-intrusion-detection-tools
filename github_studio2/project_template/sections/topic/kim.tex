
\subsubsection{OSSEC}

OSSEC capabilities is supported with the most used operating systems for businesses, which run on Linux, Windows or Mac OS. The system is free to use and this makes it ideal for small businesses, that might have a tight budget to finance their investments. The features that OSSEC home page (ossec.net/about) is claiming to have is Log based Intrusion Detecion(LIDs), Rootkit and Malware Detection, Active Response, Compliance Auditing, File Integrity Monitoring (FIM) and System Inventory.\\

According to (Reznik) Suricata is the second most popular IDS tools used, only bypassed by SNORT in popularity. Containing features like hardware acceleration and multithreading to perform spesific functions more effectivley.

Suricata has according to the (techfunnel article) all of the mentioned capabilites of OSSEC and is claimed to be fast and could also detect advanced threats. This capabilities are also supported by a literature review paper on IDS (A Comprehensive Systematic Literature Review on Intrusion Detection Systems) where the advantages of Suricata are mentioned along with additional benefits like filtered events and alarms, automatic protocol detection, application layer data collection to name a few. The downsides however are something that a small business should take into consideration as the CPU usage is high, and it has a smaller supported community compared to SNORT. The installation process is also mentioned as complex, and the need for special competance in this area for a small business may be needed.\\