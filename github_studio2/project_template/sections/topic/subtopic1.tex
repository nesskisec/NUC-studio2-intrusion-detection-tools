\subsubsection{OSSEC}

OSSEC capabilities is supported with the most used operating systems for businesses, which run on Linux, Windows or Mac OS. The system is free to use and this makes it ideal for small businesses, that might have a tight budget to finance their investments. The features that OSSEC home page (ossec.net/about) is claiming to have is Log based Intrusion Detecion(LIDs), Rootkit and Malware Detection, Active Response, Compliance Auditing, File Integrity Monitoring (FIM) and System Inventory.\\

According to (Reznik) Suricata is the second most popular IDS tools used, only bypassed by SNORT in popularity. Containing features like hardware acceleration and multithreading to perform spesific functions more effectivley.

Suricata has according to the (techfunnel article) all of the mentioned capabilites of OSSEC and is claimed to be fast and could also detect advanced threats. This capabilities are also supported by a literature review paper on IDS (A Comprehensive Systematic Literature Review on Intrusion Detection Systems) where the advantages of Suricata are mentioned along with additional benefits like filtered events and alarms, automatic protocol detection, application layer data collection to name a few. The downsides however are something that a small business should take into consideration as the CPU usage is high, and it has a smaller supported community compared to SNORT. The installation process is also mentioned as complex, and the need for special competance in this area for a small business may be needed.\\

\subsubsection{SNORT}

\begin{notes}[Individual subtopic - SNORT]

    Work on-going:

    \begin{itemize}
        \item Gathering information about implementation and deployment examples that can be applied to SMBs.
        \item Gathering concret implementation example where SNORT is used in conjucntion with ML and DL to address polymorphic and metamorphic malware.\\
    \end{itemize}
    
\end{notes}

\subsubsection{Zeek}

According to [1] Zeek (previously known as Bro) is an intrusion detection system that differs from others in that it focuses on network analysis. Unlike rules-based engines, which are meant to identify exceptions, Zeek searches for dangers and generates warnings. Zeek is an open-source, passive network traffic analyser. Zeek is widely used by operators as a network security monitor (NSM) [2] to aid in the investigation of suspicious or malicious behaviour. Zeek also provides support for a broad variety of traffic analysis activities outside of the security area, such as performance assessment and troubleshooting.
Extracting data from HTTP sessions, detecting malware by interacting with external registries, reporting vulnerable versions of software visible on the network, recognising popular online apps, detecting SSH brute-forcing, checking SSL certificate chains, and much more are all included into Zeek.
\\

{\bfseries{Pros:}}

\begin{itemize}
    \item Zeek is a traffic analysis tool that is completely flexible and expandable. To express any analytic job, Zeek offers a domain-specific, Turing-complete scripting language.
    \item A low-cost alternative to pricey proprietary technologies, Zeek operates on commodity hardware. Zeek is a network monitoring tool that goes well beyond the capabilities of most other tools, which are often confined to a narrow collection of pre-programmed analytic activities.
    \item As a platform for gathering and analysing network data, Zeek is best suited to the task at hand.
    \item The transaction data provided by Zeek is the company's biggest selling point. This means that if a network interface is being monitored, Zeek will create a collection of high-fidelity, highly annotated transaction logs by default. In a judgment-free, policy-neutral way, these logs explain the protocols and activities on the network.
    \\
\end{itemize}

{\bfseries{Cons:}}

\begin{itemize}
    \item ZeeK/Bro's deep packet inspection consumes a lot of resources, which is a drawback if you're looking for flexibility. In terms of threat intelligence, Snort and Suricata are the two most popular options. The community is continuously attempting to make Bro more user-friendly because of this.
    \\
\end{itemize}

\subsubsection{IBM QRADAR}

IBM has developed QRadar, a tool for handling security issues [3]. Additionally, it can analyse data from a wide range of sources (router/firewall/application/folder) and store and manage data, as well as uncover vulnerabilities and information about risks to data security. Additionally, QRadar has a variety of monitoring functions that look for changes in user or network activity that might suggest an attack or a policy violation. QRadar may send notifications to the appropriate recipient, for example by e-mail, informing them of the occurrence of a violation or attack.
\\

{\bfseries{Pros:}}

\begin{itemize}
    \item Streamlines the identification and prioritization of threats in the IT infrastructure.
    \item Simplifies alert processing so that security analysts may concentrate their investigations on a smaller number of high-probability threats.
    \item Improves threat management by generating comprehensive data access and user activity reports for each user.
    \item The ability to work in both on-premises and in the cloud
    \item Complies with regulations by generating thorough reports on data access and user activities.
    \item Security intelligence solutions may be provided cost-effectively by managed service providers using multi-tenancy and a master console.
    \\
\end{itemize}

{\bfseries{Cons:}}

Based of user reviews the cons of IBM Qradar Siem are mentioned below.\\
\begin{itemize}
    \item The product is quite sluggish since it was designed using outdated technologies. Windows log collection is very time-consuming and antiquated."
    \item To have a system that correctly warns you when an attack is taking place, you can't just click a couple buttons."vBecause to this, IBM QRadar couldn't be used for correlation.
    
\end{itemize}
