\documentclass[conference]{IEEEtran}
\IEEEoverridecommandlockouts

\usepackage{cite}
\usepackage{amsmath,amssymb,amsfonts}
\usepackage{algorithmic}
\usepackage{graphicx}
\usepackage{textcomp}
\usepackage{xcolor}
\def\BibTeX{{\rm B\kern-.05em{\sc i\kern-.025em b}\kern-.08em
    T\kern-.1667em\lower.7ex\hbox{E}\kern-.125emX}}

% ------ START Custom Section For "usepackage" ------

\usepackage{hyperref}
\usepackage{epigraph}
\usepackage{csquotes}
\usepackage{tikz}
\usepackage{xcolor}
\usepackage{svg}
\usepackage{float} % Enables you to force a "floating object" like the environments 'figure' and 'table' inline with the text.
\usepackage{morefloats} % Enalbes LaTeX to exceed the maximum 18 floats in a document.
\usepackage{nameref}
\usepackage{wrapfig}
\usepackage{subcaption} % Enable pictures to be aligned side by side. NOT COMPATIBLE with adjustbox.
\usepackage{adjustbox}
\usepackage{cancel}
\usepackage{ulem}
\graphicspath{ {./images/} }

% packages for typesetting source-code
\usepackage{listings} % Required for inserting code snippets

% Optional packages
\usepackage{blindtext} % Sample text, false content, Lorem Ipsum type of random generated text.

% ------ END Custom Section For "usepackage" ------


% ------ START Custom Section For environment settings ------

\newenvironment{notes}[1][\unskip]
{
\par
\noindent
\textcolor{blue}{\bfseries{Notes - } #1:}
\\ \color{blue}}
{}

% \newenvironment{notes}[1]
%     {\begin{center}
%     #1\\[1ex]
%     \begin{tabular}{|p{50mm}|}
%     \hline\\
%     }
%     { 
%     \\\\\hline
%     \end{tabular} 
%     \end{center}
%     }

\newenvironment{followup}[1][\unskip]{%
\par
\noindent
\textcolor{red}{\bfseries{Follow-up - } #1!!!}
\\ \color{red}}
{}

\newenvironment{question}[1][\unskip]{%
\par
\noindent
\textcolor{orange}{\bfseries{Question - } #1???}
\\ \color{orange}}
{}

% Typesetting code snippets
% The code snippet typesetting below is based on https://www.latextemplates.com/template/code-snippet
% Original Author: This template was created for LaTeXTemplates.com by vel@latextemplates.com
\definecolor{DarkGreen}{rgb}{0.0,0.4,0.0} % Comment color
\definecolor{highlight}{RGB}{204,255,229} % Code highlight color
\definecolor{Gray}{RGB}{224,224,224}
\definecolor{Purple}{RGB}{255,0,255}
\definecolor{Blue}{RGB}{0,0,255}

% ------ END Custom Section For environment settings ------



\begin{document}

\title{A Comparison of Intrusion Detection Tools and why these are important for SMB's\\}

\author{\IEEEauthorblockN{Kim André Næss, Dale Peregrino Bada, Muhammad Javed Iqbal}
\IEEEauthorblockA{\textit{Cyber Security} \\
\textit{Noroff University College}\\
Kristiansand, Oslo, Norway \\
kim.naess@stud.noroff.no, dale.bada@stud.noroff.no, muhammad.iqbal@stud.noroff.no}
}


\maketitle

\begin{abstract}

In this state-of-art review paper, we will examine the latest Intrusion Detection Systems (IDS) and Intrusion Prevention Systems (IPS) applicable for Small and Midsize Busnisess (SMB), where SMB are as defined by Gartner\cite{gartner}.

This review has 3 key objectives. Firstly, we will define and identify what kind of network security threats are most prevalent and most urgent for SMBs' to address. Secondly, we will identify what open source software are available for SMBs' to manage and mitigate identified threats. Then in the third final part, we will review a select number of these open source software and assess how well each system protects SMBs' from the identified network threats.

Each tools' features, capabilities, advantages and limitations will be compared. How each system fare with regards to managebility, cost effectiveness and return of investment also will be taken into consideration before we conclude which system will be crowned with a recommendation as IDS/IPS for SMBs'. 

\end{abstract}

% \begin{abstract}
% Lorem ipsum dolor sit amet mollit sunt duis velit non aliquip in labore minim. Proident mollit pariatur nisi id minim incididunt excepteur voluptate ad fugiat sunt. Anim sint in reprehenderit exercitation laborum ex laboris quis anim consequat proident pariatur dolor labore in mollit pariatur velit laboris id aute. Culpa consequat aute dolore eu non ullamco culpa sit ipsum labore
% \end{abstract}

\begin{IEEEkeywords}

Intrusion Detection Tools (IDS), Small Business, Malware, Ransomware, Machine Learning

\end{IEEEkeywords}

\section{Introduction}
Eiusmod consequat Lorem occaecat ut adipisicing in Lorem. Ad ullamco labore, ipsum quis duis et in excepteur veniam nisi cupidatat veniam deserunt dolore sit consequat et excepteur tempor esse amet. Lorem voluptate nostrud aliquip incididunt nisi ipsum sint ad irure officia nulla deserunt incididunt ullamco dolor. Velit aute commodo adipisicing, eu aute amet quis ea cupidatat aliqua Lorem do pariatur quis dolor. Eu labore esse ullamco Lorem quis reprehenderit enim nisi do consequat consectetur officia ipsum irure enim ad laboris reprehenderit pariatur cupidatat enim laboris. Minim do ullamco aute sunt ut veniam.

Quis anim pariatur proident, et laborum velit aute ea aliqua ullamco irure cupidatat non proident magna esse. Laboris eiusmod elit do cupidatat incididunt cupidatat ad non. Dolore officia aliquip ut sit adipisicing voluptate aute adipisicing. Consequat excepteur irure pariatur Lorem laborum est eiusmod, velit ex ipsum non sit. Duis tempor, amet qui dolor ex et qui occaecat culpa. Consequat amet sit aute occaecat sint deserunt, anim aute pariatur dolor. Voluptate irure dolor eu culpa dolore irure dolore veniam.

\section{Reviewing IDS Systems for Small and Mid-sized Businesses}
\begin{notes}[work in progress]

    Section content being assessed.
    
\end{notes}

\subsection{Subtopic1}
\subsubsection{OSSEC}

OSSEC capabilities is supported with the most used operating systems for businesses, which run on Linux, Windows or Mac OS. The system is free to use and this makes it ideal for small businesses, that might have a tight budget to finance their investments. The features that OSSEC home page (ossec.net/about) is claiming to have is Log based Intrusion Detecion(LIDs), Rootkit and Malware Detection, Active Response, Compliance Auditing, File Integrity Monitoring (FIM) and System Inventory.\\

According to (Reznik) Suricata is the second most popular IDS tools used, only bypassed by SNORT in popularity. Containing features like hardware acceleration and multithreading to perform spesific functions more effectivley.

Suricata has according to the (techfunnel article) all of the mentioned capabilites of OSSEC and is claimed to be fast and could also detect advanced threats. This capabilities are also supported by a literature review paper on IDS (A Comprehensive Systematic Literature Review on Intrusion Detection Systems) where the advantages of Suricata are mentioned along with additional benefits like filtered events and alarms, automatic protocol detection, application layer data collection to name a few. The downsides however are something that a small business should take into consideration as the CPU usage is high, and it has a smaller supported community compared to SNORT. The installation process is also mentioned as complex, and the need for special competance in this area for a small business may be needed.\\

\subsubsection{SNORT}

\begin{notes}[Individual subtopic - SNORT]

    Work on-going:

    \begin{itemize}
        \item Gathering information about implementation and deployment examples that can be applied to SMBs.
        \item Gathering concret implementation example where SNORT is used in conjucntion with ML and DL to address polymorphic and metamorphic malware.\\
    \end{itemize}
    
\end{notes}

\subsubsection{Zeek}

According to [1] Zeek (previously known as Bro) is an intrusion detection system that differs from others in that it focuses on network analysis. Unlike rules-based engines, which are meant to identify exceptions, Zeek searches for dangers and generates warnings. Zeek is an open-source, passive network traffic analyser. Zeek is widely used by operators as a network security monitor (NSM) [2] to aid in the investigation of suspicious or malicious behaviour. Zeek also provides support for a broad variety of traffic analysis activities outside of the security area, such as performance assessment and troubleshooting.
Extracting data from HTTP sessions, detecting malware by interacting with external registries, reporting vulnerable versions of software visible on the network, recognising popular online apps, detecting SSH brute-forcing, checking SSL certificate chains, and much more are all included into Zeek.
\\

{\bfseries{Pros:}}

\begin{itemize}
    \item Zeek is a traffic analysis tool that is completely flexible and expandable. To express any analytic job, Zeek offers a domain-specific, Turing-complete scripting language.
    \item A low-cost alternative to pricey proprietary technologies, Zeek operates on commodity hardware. Zeek is a network monitoring tool that goes well beyond the capabilities of most other tools, which are often confined to a narrow collection of pre-programmed analytic activities.
    \item As a platform for gathering and analysing network data, Zeek is best suited to the task at hand.
    \item The transaction data provided by Zeek is the company's biggest selling point. This means that if a network interface is being monitored, Zeek will create a collection of high-fidelity, highly annotated transaction logs by default. In a judgment-free, policy-neutral way, these logs explain the protocols and activities on the network.
    \\
\end{itemize}

{\bfseries{Cons:}}

\begin{itemize}
    \item ZeeK/Bro's deep packet inspection consumes a lot of resources, which is a drawback if you're looking for flexibility. In terms of threat intelligence, Snort and Suricata are the two most popular options. The community is continuously attempting to make Bro more user-friendly because of this.
    \\
\end{itemize}

\subsubsection{IBM QRADAR}

IBM has developed QRadar, a tool for handling security issues [3]. Additionally, it can analyse data from a wide range of sources (router/firewall/application/folder) and store and manage data, as well as uncover vulnerabilities and information about risks to data security. Additionally, QRadar has a variety of monitoring functions that look for changes in user or network activity that might suggest an attack or a policy violation. QRadar may send notifications to the appropriate recipient, for example by e-mail, informing them of the occurrence of a violation or attack.
\\

{\bfseries{Pros:}}

\begin{itemize}
    \item Streamlines the identification and prioritization of threats in the IT infrastructure.
    \item Simplifies alert processing so that security analysts may concentrate their investigations on a smaller number of high-probability threats.
    \item Improves threat management by generating comprehensive data access and user activity reports for each user.
    \item The ability to work in both on-premises and in the cloud
    \item Complies with regulations by generating thorough reports on data access and user activities.
    \item Security intelligence solutions may be provided cost-effectively by managed service providers using multi-tenancy and a master console.
    \\
\end{itemize}

{\bfseries{Cons:}}

Based of user reviews the cons of IBM Qradar Siem are mentioned below.\\
\begin{itemize}
    \item The product is quite sluggish since it was designed using outdated technologies. Windows log collection is very time-consuming and antiquated."
    \item To have a system that correctly warns you when an attack is taking place, you can't just click a couple buttons."vBecause to this, IBM QRadar couldn't be used for correlation.
    
\end{itemize}


\subsection{Subtopic2}
\begin{notes}[work in progress]

    \begin{itemize}
        \item \sout{Which is the best suited IDS system for SMB companies}
        \begin{itemize}
            \item \sout{How easy are the IDS systems to by-pass?}
            \item What are the minimum HW required for deplyoment? How hard or easy are the IDS to install, run and manage?
            \item \sout{How effective are the IDS to detect/safeguard against ransomeware/malware?}
        \end{itemize}

    \end{itemize}
    
    Section content being assessed.
    
\end{notes}

\subsection{Subtopic3}
Lorem ipsum dolor sit amet est velit Lorem adipisicing consectetur irure qui, consequat. Excepteur adipisicing occaecat ex non sint deserunt proident id minim eiusmod excepteur amet pariatur. Id culpa id eiusmod veniam irure enim mollit eiusmod ex. Nostrud ut qui non mollit ullamco, velit voluptate elit reprehenderit ut Lorem mollit adipisicing consequat laborum id. Et quis aliqua cupidatat eu eiusmod irure mollit dolor aliqua ut aliquip magna. Ad minim culpa eu labore aliqua enim incididunt occaecat ad aliquip velit ad incididunt excepteur officia aute, ex.

\newpage

\section{Ethical Issues}
Lorem ipsum dolor sit amet est velit Lorem adipisicing consectetur irure qui, consequat. Excepteur adipisicing occaecat ex non sint deserunt proident id minim eiusmod excepteur amet pariatur. Id culpa id eiusmod veniam irure enim mollit eiusmod ex. Nostrud ut qui non mollit ullamco, velit voluptate elit reprehenderit ut Lorem mollit adipisicing consequat laborum id. Et quis aliqua cupidatat eu eiusmod irure mollit dolor aliqua ut aliquip magna. Ad minim culpa eu labore aliqua enim incididunt occaecat ad aliquip velit ad incididunt excepteur officia aute, ex.

Veniam culpa quis incididunt anim enim aliquip ut reprehenderit voluptate nostrud aute nostrud. Commodo, consectetur in amet mollit anim cillum do cillum excepteur tempor aute amet. Cillum ut reprehenderit Lorem nisi veniam do veniam ea ipsum, quis culpa Lorem velit in ea laboris ut aliqua officia. Cillum sit duis veniam officia minim exercitation voluptate anim non, id aliquip dolor non dolor nostrud dolore eu veniam. Fugiat culpa elit ea et non esse.

Do incididunt exercitation minim irure exercitation fugiat officia duis est nostrud minim sint ipsum officia eiusmod mollit consequat aliqua ea occaecat. Dolor Lorem amet pariatur deserunt enim ut sit elit fugiat reprehenderit commodo tempor nostrud ipsum officia ea occaecat. Ullamco sunt officia irure deserunt, consectetur velit ut. Occaecat aliqua proident commodo pariatur ipsum eu aliquip sunt adipisicing aliquip amet nulla reprehenderit nulla ea laboris est dolore. Commodo laborum consequat sunt enim aliquip non aliquip amet nostrud dolore ad proident, veniam ex et fugiat dolor exercitation magna exercitation enim. Anim dolore exercitation irure officia dolor aliquip consequat culpa esse anim velit consequat aute nostrud occaecat pariatur veniam commodo duis sunt reprehenderit voluptate veniam nostrud.

\newpage

\section{Summary}
\subsubsection{Reliable and Well Supported And Well Supported}



Snort's 3 main features or mode of operation, as ofted cited in many Snort tutorial blog post entries, how-to videos on YouTube and Wikipedia are; Packet Sniffer, Packet Logger and Network Intrusion Detection System mode\cite{Wikipedia2022_Snort}. However, Cisco Systems themselves often refer to Snort as an Intrusion Prevention System in their documentation\cite{Ciscco_Snort3_Configuration_Guide_v7} and Cisco Talos Intelligence Group's YouTube video\cite{TIG2020_Snort_101}. As a packet sniffer, it is capable to capture, interpret and process packets akin to Wireshark and tcpdump. As a packet logger Snort can capture network packets (without analyzing) and store them for further analysis.As an Intrusion Detector, Snort has the ability to log events, trigger alerts and other systems it can integrate with like SIEMs. As an Intrusion Prevention System, it can drop packets from the network or trigger other systems that are able to execute preventive measures.\\

\subsubsection{IDS/IPS - Installation}

Lets establish a scenario where we want to address the issue of general intrusion and more specifically malware and ransomware. In this scenario we want to alert malicious network intrusion. At the same time detect malware prevent file extraction from a specific on-site file server.

One simple way to achieve this objective is to implement Snort as an inline NIDS/NIPS inside the SMB network, right behind a firewall. A dedicated AT\&T OSSIM server as SIEM alert and monitoring dashboard. And a dedicated server to run malware analysis on automated with Python scripts and libraries.

The Snort NIDS/NIPS can be configured to perform packet logging regularly with specific interval. This is necessary to capture data that can be used for malware analysis. A static analysis, automated with Python scripts utilizing OpenSource libraries for Machine and Deep Learning, read Windows Portable Executables (PE) etc.

There are different methods to apply Machine Learning to perform malware analysis as described in Alessandro Parisi's "Hands-On Artificial Intelligence for CyberSecurity"\cite{Parisi2019} while A YouTube video by Motasem Hamdan describes basic installation of Snort\cite{YouTube2022_HamdanM}\\

\section{Conclusion}
Security best-practice rely on a holistic and layered approach...

\begin{followup}[to-do]
    \begin{itemize}
        \item Provide evidence for security best-practices to further elaborate and support the statement.
        \item Provide concrete examples where IDS and IPS are ineffective against malware and ransomware.
    \end{itemize}
\end{followup}

With all the limitations inherent to IDS and IPS taken into consideration. The most effective, cost effective and manageable systems is...

\begin{followup}[to-do]
    \begin{itemize}
        \item Insert name of IDS/IPS here when all the review has been conducted.
    \end{itemize}
\end{followup}

\newpage

\section*{Appendix A:}
% Items are shown in "main.bib" file\\

% \noindent An article \cite{anarticle} - The required fields here are author, title, journal, and year; all others can be omitted if you do not have the corresponding information. Note as well that author names should be separated by “and”.\\
% A book \cite{abook} - The required fields are author, title, publisher, and year.\\
% A series \cite{bookseries} - Series are cited as books with an additional field.\\
% Someone's thesis \cite{thesis} - The required fields are author, title, publisher, and year. You may also
% cite master’s theses using the \textit{mastersthesis} entry type.\\
% Some technical report \cite{report} - The required fields are author, title, publisher, and year.\\
% A collection \cite{collection} - The required fields are author, title (of the article within the book), book-
% title (name of the book containing the article), and year.\\
% Visited website \cite{website} - For websites, note the specific formatting of the howpublished field. Be
% sure to include the date of access as shown. Also, keep in mind that certain
% characters common in urls, such as underscores, must be escaped (preceded
% by a backslash, ex.\\
% Accepted for publication \cite{acceptedpub} - To show that an article has been accepted for publication but not yet
% published, include the line note = {(in press)}.\\
% Submitted for publication \cite{unpub} - cited as published with an additional field.\\
% Not published \cite{notpub} - If a manuscript has not yet been published or submitted for publication,
% you may still cite it. Include a note to this effect.\\
% Conversation \cite{conv} - Occasionally, an important item that does not exist in the literature will
% be discussed with your mentor. When this happens, you may cite the conversation with your mentor.

\bibliographystyle{IEEEtran}
\bibliography{main}

\newpage

\section*{Appendix B:}

\subsection{IDS Feature Comparison}

\begin{table}[!ht]
    \centering
    \begin{tabular}{|p{1.4cm}|p{1.4cm}|p{1.4cm}|p{1.4cm}|p{1.4cm}|}
    \hline
        Software & Zeek & OSSEC & Suricata & Snort \\ \hline
        NIDS & X & ~ & X & X \\ \hline
        NIPS & ~ & ~ & X & X \\ \hline
        HIDS & ~ & X & ~ & X \\ \hline
        HIPS & ~ & ~ & ~ & X \\ \hline
        GUI & ~ & ~ & ~ & ~ \\ \hline
        CLI & X & ~ & ~ & X \\ \hline
        SIEM & ~ & X & ~ & Supported, Syslog \\ \hline
        Packet logg/Capture & X & ~ & ~ & X \\ \hline
        Packet sniffer/filter & ~ & ~ & ~ & X \\ \hline
        Signature Detection & X & ~ & ~ & X \\ \hline
        Anomaly Detection & X & X & ~ & X \\ \hline
        Malware Detection & X & ~ & X & Supported \\ \hline
        Other features & Misuse detection, behavioral analysis & Rootkit and trojan detection & PCAP analysis, Multi-Threaded, Application Layer Logging & Malware detection is Prepocessor Plugin \\ \hline
        OS Platform Support & OSX, Linux, Windows & OSX, Linux, Windows & OSX, Linux, Windows & ARM, OSX, Linux, Windows, Cisco IOS-XE \\ \hline
        Architecture Support & AMD X86, Intel X86 & AMD X86, Intel X86 & AMD X86, Intel X86 & ARM, AMD X86, Intel X86, Cisco ASIC \\ \hline
        OpenSource Software & X & X & X & X \\ \hline
        Business, Enterprise Edition & ~ & ~ & ~ & X \\ \hline
    \end{tabular}
    \caption{Comparison Table}
    \label{Feature_Comparison}
\end{table}
\newpage
\subsection{Zeek Pros and Cons} \label{prosncons}

{\bfseries{Pros:}}

\begin{itemize}
    \item Zeek is a traffic analysis tool that is completely flexible and expandable. To express any analytic job, Zeek offers a domain-specific, Turing-complete scripting language.
    \item A low-cost alternative to pricey proprietary technologies, Zeek operates on commodity hardware. Zeek is a network monitoring tool that goes well beyond the capabilities of most other tools, which are often confined to a narrow collection of pre-programmed analytic activities.
    \item As a platform for gathering and analysing network data, Zeek is best suited to the task at hand.
    \item The transaction data provided by Zeek is the company's biggest selling point. This means that if a network interface is being monitored, Zeek will create a collection of high-fidelity, highly annotated transaction logs by default. In a judgment-free, policy-neutral way, these logs explain the protocols and activities on the network.\\

\end{itemize}

{\bfseries{Cons:}}

\begin{itemize}
    \item ZeeK/Bro's deep packet inspection consumes a lot of resources, which is a drawback if you're looking for flexibility. In terms of threat intelligence, Snort and Suricata are the two most popular options. The community is continuously attempting to make Bro more user-friendly because of this.
\end{itemize}


\newpage
\newpage

%%% DO NOT  modify below.
\newcommand{\noroffcount}[1]{%
\immediate\write18{texcount -v0 -q -total  -sum -merge -q #1.tex > #1-words.noroff }%
  \input{#1-words.noroff} %
}
%
%Words with Refs:\quickwordcount{main}
%
%Words Without refs: \quickwordcountnoref{main}

\newcommand{\NUCwordcount}[1]{
    \section*{Word count metrics}
    \framebox{%
    \begin{minipage}{0.95\textwidth}
    \textbf{NUC Studio2 Word Count}:\\
    \noroffcount{#1}
    NOTE: References are excluded.
    \end{minipage}}
}

\newpage
% \NUCwordcount{main}
\end{document}